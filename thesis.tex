\documentclass[openany, 10pt]{uithesis}
\usepackage{graphicx}
\usepackage{lipsum}

\addbibresource{biblio.bib}
\nocite{*}

\title{An Exploration of the Burrito Sandwich Paradox}
\author{Herky T. Hawk}
\date{April, 2021}

\ThesisSupervisor{Dr. Professorson}
\CommitteeMember{Another Professor}
\CommitteeMember{A Faculty Member}
\CommitteeMember{The Janitor}

\begin{document}

\titlepage

\frontmatter


\begin{publicAbstract}
  \lipsum[1-2]
\end{publicAbstract}

\begin{abstract}
  \lipsum[1-2]
\end{abstract}

\tableofcontents*

\listoffigures

\listoftables

\mainmatter

\chapter{Background Material}


In this chapter I introduce all the backgronud material needed to understand the
rest of the things.

\section{Man I don't know}
\begin{figure}
  \centering
  \label{fig:one}
  \caption{A very cool letter ``A''}
\includegraphics{example-image-a}
\end{figure}
\lipsum[1-5]
\begin{figure}
  \centering
  \label{fig:two}
  \caption{A very cool letter ``B''}
\includegraphics{example-image-b}
\end{figure}
\lipsum[5-10]


\section{Just testing sections}
\begin{figure}
  \centering
  \label{fig:three}
  \caption{A very cool letter ``C''}
\includegraphics{example-image-c}
\end{figure}
\lipsum[11-20]

\chapter{New Stuff}
\lipsum[21-40]

\chapter{Results}
\begin{table}
  \centering
  \label{table:one}
\begin{tabular}{ c c c }
 cell1 & cell2 & cell3 \\ \hline
 cell4 & cell5 & cell6 \\ \hline
 cell7 & cell8 & cell9    
\end{tabular}
  \caption{A very cool table with no meaning}
\end{table}
\lipsum[41-60]

\backmatter

\printbibliography

\end{document}
