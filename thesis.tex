\documentclass[letterpaper, openany, 12pt, oneside]{uithesis}
\usepackage{graphicx}
\usepackage{lipsum}

\addbibresource{biblio.bib}
\nocite{*}

\title{An Exploration of the Burrito Sandwich Paradox}
\author{Herky T. Hawk}
\date{May, 2021}

\ThesisSupervisor{Dr. Professorson}
\CommitteeMember{Another Professor}
\CommitteeMember{A Faculty Member}
\CommitteeMember{The Janitor}
\CommitteeMember{Lucky No. 5}

\begin{document}

\titlepage

\copyrightPage

\frontmatter

\acknowledgments
This is where all the helpful people get acknowledged and thanked. And now for filler.

\lipsum[1]

\begin{publicAbstract}
	\lipsum[1-2]
\end{publicAbstract}

\begin{abstract}
	\lipsum[1-2]
\end{abstract}

\tableofcontents*

\listoffigures

\listoftables

\mainmatter

\chapter{Background Material}

In this chapter I introduce all the backgronud material needed to understand the
rest of the things.

\section{Man I don't know}
\begin{figure}
	\centering
	\label{fig:one}
	\caption{A very cool letter ``A''}
	\includegraphics[width=0.8\linewidth]{example-image-a}
\end{figure}
\lipsum[1-5]
\begin{figure}
	\centering
	\label{fig:two}
	\caption{A very cool letter ``B''}
	\includegraphics[width=0.8\linewidth]{example-image-b}
\end{figure}
\lipsum[6-10]

\section{Just testing sections}
\begin{figure}[!htb]
	\centering
	\label{fig:three}
	\caption{A very cool letter ``C''}
	\includegraphics[width=0.8\linewidth]{example-image-c}
\end{figure}
\lipsum[11-15]

\subsection{How do subsections look?}
\lipsum[75]


\chapter{New Stuff}
\lipsum[21-40]

\section{A section about said new stuff}
\lipsum[41-45]

\subsection{Oh look, another rather short subsection}
\begin{figure}[!htb]
	\centering
	\includegraphics[width=0.8\linewidth]{example-image}
	\caption{A very cool image}%
	\label{fig:four}
\end{figure}
\lipsum[46-47]

\chapter{Results}
\begin{table}
	\centering
	\label{table:one}
	\begin{tabular}{ c c c }
		cell1 & cell2 & cell3 \\ \hline
		cell4 & cell5 & cell6 \\ \hline
		cell7 & cell8 & cell9
	\end{tabular}
	\caption{A very cool table with no meaning}
\end{table}
\lipsum[48-61]

\backmatter

\printbibliography

\end{document}
